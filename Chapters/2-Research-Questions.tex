\chapter{Potential Axis of research for Green FabLab Project}

\section{Elements of research to explore}

\begin{small}
	
	Table \ref{Research.Perspectives.Technical} presents some potential elements	for research:
	
	
	\begin{longtabu} to \linewidth [H] { X[0.5,l]  X[0.5,l]  X[1.4, l]  X[0.5, l]   X[1,l]   }
		
		\caption{Possible element to study / resolve} \\
		
		\toprule
		\textbf{Theme}	& \textbf{Focus} & \textbf{Goal} & \textbf{Based on the work of} & \textbf{Resources we need}  \\ 		
		\midrule
		\endfirsthead
		
		
		%	\multicolumn{9}{c}{{\bfseries \tablename\ \thetable{} -- continued from previous page}} \\[0.5mm]
		%	\toprule
		%	& \textbf{Research focus} & \textbf{Author} & \textbf{AM Process }& \textbf{Material Type} & \multicolumn{4}{c}{Sustainable dimensions}  \\ 
		%	\midrule
		%	\endhead
		%	
		%	
		%	\midrule 
		%	\multicolumn{6}{|l|}{DP= Discrete Particle; MM= Molten Material}& \multicolumn{3}{|l|}{ }\\
		%	\multicolumn{6}{|l|}{PM=Powder Metal: PP=Polymer Powder; Fil.= Filament}&  \multicolumn{3}{|l|}{Continued on}\\
		%	\multicolumn{6}{|l|}{Exp.= Experimental; Conc.= Conceptual}&\multicolumn{3}{|l|}{next page} \\
		%	\midrule
		%	\endfoot
		%	
		%	\midrule 
		%	\multicolumn{9}{|l|}{DP= Discrete Particle; MM= Molten Material} \\
		%	\multicolumn{9}{|l|}{PM=Powder Metal: PP=Polymer Powder; Fil.= Filament}  \\
		%	\multicolumn{9}{|l|}{Exp.= Experimental; Conc.= Conceptual}  \\
		%	\bottomrule
		%	\endlastfoot
		
		Technical & Material & Index of \textbf{\textit{Printability}} & Antonio &  \\ 	\midrule
		Technical & Material & Toxicology in 3D Printing / LCA & Antonio & LIST - Sebastian Zinck \\ 	\midrule
		Technical & Extrusion at LF2L & Calibration of the extrusion process using material of LF2L & CESI group & Repairing the Machine.  \\ \midrule
		Logistical & Collection & Continuation of the mathematical model from Pavlo & Pavlo & Pavlo  \\  \midrule
		Technical & Material & Test material in 3DP from LIST lab & - & Master and LIST Agreement  \\  \midrule
		Usage &  & LF2L et Fabcities : deux approches convergentes? &  & ENSGSI students \\  \midrule								
		
		\bottomrule
		\label{Research.Perspectives.Technical}
	\end{longtabu}	
\end{small}


\section{Some elements to explore}


However, some elements to explore are:

\begin{itemize}
	\item LF2L et Fabcities : deux approches convergentes?. ENSGSI
	\item \textbf{Holoprest:} a entrepreneur (Aurélien STOKY) that want to develop a manual injection machine using recycled polymer. He needs us in order to scientific base for recycling, and the use of machines. (\textcolor{blue}{Meeting Fabio September 12/2017}).
	On the other hand, we can valorise the recycled material using a nw injection process in parallel with 3DP
	
\end{itemize}

