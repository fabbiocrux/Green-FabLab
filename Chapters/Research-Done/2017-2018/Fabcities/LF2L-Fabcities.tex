

\chapter{LF2L et Fabcities : deux approches convergentes}


\section{Presentation of the subject}


\begin{tcolorbox}[
	toptitle=3mm,
	colframe=blue!25,
	colback=blue!10,
	coltitle=blue!20!black,  
	fonttitle=\bfseries,
	boxrule=1mm,
	segmentation style={line width=0.5mm, blue!25},
	adjusted title= LF2L et Fabcities : deux approches convergentes ? ]

	\begin{description}
		\item[Partner:] LF2L 	- Lorraine Fab Living Lab \href{http://www.lf2l.fr}{LF2L}			
		\item[LF2L tutor:] Laurent Dupont 
		\item[ENSGSI tutor:] Hakim Boudaoud, Fabio Cruz, Mauricio Camargo 

	\end{description}



	
\textbf{Presentation of the subject:}

Detroit, Barcelonne, Amsterdam, Paris……
In a few years these cities have joined the movement of the Fabcities\footnote{ \href{ http://fab.city }{fab city is a new urban model for locally productive and globally connected self sufficient cities}
}, which refers to "an urban model that allows a more rational and green local production, while connecting its cities to each other on a global scale".

For its part, the LF2L is carrying out many actions that are close to this vision: improving the living conditions of citizens, co-designing sustainable neighborhoods, reflection on local production and recycling and sustainable ...
In July 2018, there will take place the  Fab14, the worldwide network event that supports these initiatives.

The objective of this project is to participate in the preparation of LF2L's participation in this event. And at the same time, to reflect on the similarities and differences and how the LF2L can integrate this notion so that the Nancy Metropolis becomes a FabCity.


In collaboration with the FabCities network, the work of the project group should make it possible to:

	\begin{itemize}
		\item Conduct a benchmarking on the concept of short circuit recycling (national and international level).
		\item Identify the best practices in the "Fabcities": organization, technologies, economic model.
		\item Propose solutions adapted to the local environment of the LF2L and the territory of the  Nancy Metropolis (analysis of the needs of the local ecosystem).

	\end{itemize}

\tcbline
	\begin{description}
		\item[Start date:] Octobre 2017
		\item[Equipe projet:] 4 students  de  $2^{nd}$ year.
	\end{description}

\end{tcolorbox}
\newpage

