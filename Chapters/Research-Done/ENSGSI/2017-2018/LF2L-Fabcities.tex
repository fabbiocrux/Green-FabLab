\subsection{Presentation of the subject Probando}



\begin{description}
	\item[Title:] LF2L et Fabcities : deux approches convergentes ?  
	\item[Partner:] LF2L 	- Lorraine Fab Living Lab \href{http://www.lf2l.fr}{LF2L}			

	\item[LF2L tutor:] Laurent Dupont 
	\item[ENSGSI tutor:] description
\end{description}

Concerning the first point of only virgin material was considered, Antonio Araujo proposed a methodology that defines in a safe way the technical feasibility of recycling a thermoplastic material for 3D printing, using the extrusion process and the additive manufacturing FDM technology.

As a context for this research, four types of materials from the association Broplast were considered (figure \ref{Context.Antonio})

%\begin{figure}[H]
%	\centering
%	\includegraphics[width=0.8\textwidth]{Figures/Antonio/Context.png}
%	\caption{Four materials from Broplast Association}
%	\label{Context.Antonio}
%\end{figure}


%Figure \ref{Methodology.Antonio} shows the proposed methodology.

%\begin{figure}[H]
%	\centering
%	
%	\begin{subfigure}{0.55\textwidth}
%		\includegraphics[width=\textwidth]{Figures/Antonio/Methodology.pdf}
%		\caption{Methodological framework for recycling in 3DP.}
%		\label{Methodology.Antonio}
%	\end{subfigure}
%	\hfill
%	\begin{subfigure}{0.4\textwidth}
%		\includegraphics[width=0.8\textwidth]{Figures/Antonio/Final-Results.png} 
%		\caption{Operational methodology for recycling in 3DP.}
%		\label{Results.Antonio}		
%	\end{subfigure}
%
%	
%	\caption{Methodology and results for screening aporoach for recycling process}
%	\label{Research.Antonio}
%\end{figure}